\documentclass[autodetect-engine,dvipdfmx]{jsarticle}

\usepackage{okumacro}
\usepackage[top=25truemm,bottom=25truemm,left=25truemm,right=25truemm]{geometry}
\usepackage[dvipdfmx]{xcolor}
\usepackage{ascmac}
\usepackage{listings,jlisting}
\usepackage{latexltx2015}

\begin{document}

\title{ latex.ltxリーディング \\ 第7回資料 }
\author{ 東大\TeX 愛好会 }
\date{2015年10月12日}
\maketitle

\section{smash(大浦)}

\subsection{使用例}

まず,\cmd{smash}の使用例について軽く見ておこう。\cmd{smash}は,引数に与えられたテキストの「高さ・深さを0にする」というコマンドである。つまり,
\[
\int^a_b\sqrt{|x|}\,dx
\]
を大かっこでくくるような場合を考えよう。ここで,\cmd{int\^\,a\_b}を\cmd{smash}でくくるかくくらないかで次のような出力の差を生む。ちなみに一番右の例は,\cmd{left}〜\cmd{right}でくくっていないもとの大かっこの大きさで出力している。
\[
\left[\int^a_b\sqrt{|x|}\,dx\right] \left[\smash{\int^a_b}\sqrt{|x|}\,dx\right] [\int^a_b\sqrt{|x|}\,dx]
\]
インテグラルを\cmd{smash}の引数に指定しておくと,インテグラルの高さは0とみなされ,その次に一番高さが大きかった$\sqrt{|x|}$に大かっこのサイズが合うようになっている。

(ゼミで判明した内容:ちなみに,後述するとおりcolor系のパッケージをincludeしておかないと使えない模様。)

\subsection{定義}

\cmd{smash}まわりの定義をまず下に載せておく。

\latexltx
\begin{lstlisting}[firstnumber=4136]
\def\smash{%
  \relax % \relax, in case this comes first in \halign
  \ifmmode
    \expandafter\mathpalette\expandafter\mathsm@sh
  \else
    \expandafter\makesm@sh
  \fi}
\def\makesm@sh#1{%
  \setbox\z@\hbox{\color@begingroup#1\color@endgroup}\finsm@sh}
\def\mathsm@sh#1#2{%
  \setbox\z@\hbox{$\m@th#1{#2}$}\finsm@sh}
\def\finsm@sh{\ht\z@\z@ \dp\z@\z@ \box\z@}
\end{lstlisting}

この定義を理解するために,まず,数式モードの外で\cmd{smash}を使うとき,\preSub\verb|\smash{あいうえ}|\preSub というコマンドを考える。

まずこれは次のように展開される。

\texsource
\begin{lstlisting}
\relax\ifmmode\expandafter\mathpalette\expandafter\mathsm@sh\else\expandafter\makesm@sh\fi{あいうえ}
\end{lstlisting}

\cmd{relax}はそのまま読みとばされ,\cmd{ifmmode}の条件分岐に入る。いま,数式モードではないので,\cmd{else}まで飛ばされる。その次の\cmd{expandafter}によって\cmd{fi}が読まれ,条件分岐が終わったと認識される。そしてその次に\cmd{makesm@sh}が展開されていく。

つまり,

\texsource
\begin{lstlisting}
\makesm@sh{あいうえ}
\end{lstlisting}
が展開されることに相当する。
ここで\cmd{makesm@sh}の定義を代入して,次のように変換される。

\texsource
\begin{lstlisting}
\setbox\z@\hbox{\color@begingroupあいうえ\color@endgroup}\finsm@sh\end{lstlisting}

まず,\preSub\verb|\setbox\z@\hbox{...}|\preSub によって,0番のボックスレジスタにここの\cmd{hbox}を割り振る。そして\cmd{hbox}が作られる。さらに\cmd{hbox}の中を展開する。\cmd{color@begingroup}はcolor系のスタイルファイルを読み込むときに\cmd{bgroup}に置き換えられ,\cmd{color@endgroup}はcolor系のスタイルファイルをを読み込むときに,\preSub\verb|\endgraf\endgroup|\preSub になる。(ここでcolor系統のパッケージが必要になる。)ここで\cmd{endgraf}は,\texttt{latex.ltx}の438行目に次のように定義されている。

\latexltx
\begin{lstlisting}[firstnumber=438]
\let\endgraf=\par
\end{lstlisting}

また,\cmd{finsm@sh}の定義を再掲すると,次のようであった。

\latexltx
\begin{lstlisting}[firstnumber=4147]
\def\finsm@sh{\ht\z@\z@ \dp\z@\z@ \box\z@}
\end{lstlisting}

これらを代入すると,

\texsource
\begin{lstlisting}
\hbox{{あいうえ\par}}\ht\z@\z@ \dp\z@\z@ \box\z@
\end{lstlisting}
となる。つまり,0番のボックスレジスタに入れた\cmd{hbox}に対して,\cmd{ht}つまり高さ,および\cmd{dp}つまり深さを,0にする操作をするコマンドとなっている。ここが\cmd{smash}の肝となる部分といえよう。

いま,数式モードの外で\cmd{smash}を使う例を見てきた。次は,数式モード内で\cmd{smash}を使う例を見てみよう。たとえば\preSub\verb|\[\smash{1+2=3}\]|\preSub というコマンドを考える。

まずこれは次のように展開される。(\cmd{[},\cmd{]} については略した。)

\texsource
\begin{lstlisting}
\relax\ifmmode\expandafter\mathpalette\expandafter\mathsm@sh\else\expandafter\makesm@sh\fi{1+2=3}
\end{lstlisting}

\cmd{relax}はそのまま読みとばされ,\cmd{ifmmode}の条件分岐に入る。いま,数式モード内なので,次の\cmd{expandafter}が読まれる。すると,\cmd{mathpalette}を飛ばして,2つ目の\cmd{expandafter}が読まれる。これにより,\cmd{mathsm@sh}が飛ばされ,\cmd{else}が読まれる。ここで,条件分岐は終了したものと判断され,その次に\cmd{mathpalette},その次に\cmd{mathsm@sh}が展開されることとなる。つまり,

\texsource
\begin{lstlisting}
\mathpalette\mathsm@sh{1+2=3}
\end{lstlisting}
の\cmd{mathpalette}が展開されることになる。\cmd{mathpalette}は,\texttt{latex.ltx}中に次のように定義されている。

\latexltx
\begin{lstlisting}[firstnumber=4101]
\def\mathpalette#1#2{%
  \mathchoice
    {#1\displaystyle{#2}}%
    {#1\textstyle{#2}}%
    {#1\scriptstyle{#2}}%
    {#1\scriptscriptstyle{#2}}}
\end{lstlisting}

よって,次のように展開される。

\texsource
\begin{lstlisting}
  \mathchoice
    {\mathsm@sh\displaystyle{1+2=3}}%
    {\mathsm@sh\textstyle{1+2=3}}%
    {\mathsm@sh\scriptstyle{1+2=3}}%
    {\mathsm@sh\scriptscriptstyle{1+2=3}}
\end{lstlisting}

\cmd{mathchoice}は引数を四つとる\TeX プリミティブで,四つの引数の中は順に,\texttt{displaystyle},\texttt{textstyle},\texttt{scriptstyle},\texttt{scriptscriptstyle},
内でのコードを記述する。

例えば今回のように,\texttt{displaystyle}の数式モード内で\cmd{smash}を使った場合には,最初の引数内のコードが使われ, 

\texsource
\begin{lstlisting}
\mathsm@sh\displaystyle{1+2=3}
\end{lstlisting}
と展開される。


次にいよいよ\cmd{mathsm@sh}の展開だが,定義は次のようであった(再掲)。

\latexltx
\begin{lstlisting}[firstnumber=4145]
\def\mathsm@sh#1#2{%
  \setbox\z@\hbox{$\m@th#1{#2}$}\finsm@sh}
\end{lstlisting}

よって展開結果は次のようになる。

\texsource
\begin{lstlisting}
\setbox\z@\hbox{$\m@th\displaystyle{1+2=3}$}\finsm@sh
\end{lstlisting}

まず,\preSub\verb|\setbox\z@\hbox{...}|\preSub によって,0番のボックスレジスタにここの\cmd{hbox}を割り振る。そして\cmd{hbox}が作られる。さらに\cmd{hbox}の中を展開する。

\$ ハサミの効果により数式モードになっており,\cmd{m@th}によって数式モード外との境界に無駄なスペースが入らないようになっている(\cmd{m@th}については第1回Reading資料に説明がある)。そして,$1+2=3$が\texttt{displaystyle}で表示される。

さて,最後にまた,\cmd{finsm@sh}の展開である。0番のボックスレジスタに入れた\cmd{hbox}に対して,\cmd{ht}つまり高さ,および\cmd{dp}つまり深さを,0にする操作を担っていることは先に述べた通りである。

これにより,数式モード中でも\cmd{smash}を使うことができた。

%\begin{question}
%ここで,\cmd{mathsm@sh}は引数を二つ取るという事態に直面する。考えているソースでは,引数は一つしか出てこない。どうなってるんだ。
%\end{question}

%ちょっとわからなくなったのでここで筆を休めます。

\ 

さて,若干気になるのが,4137行目の記述,「\preSub\verb|\relax, in case this comes first in \halign|\preSub」だ。実際に展開した状態で\cmd{halign}の先頭で使ってみた。つまり,次のようにソースを書き,\cmd{relax}のある状態とない状態での比較を試みた。

\noindent ソース1
\texsource
\begin{lstlisting}
\halign{\relax\setbox0\hbox{{$\displaystyle\int$\par}}\ht00 \dp00 \box0#1\cr
\noalign{\hrule}
www\cr
\noalign{\hrule}}
\end{lstlisting}

\noindent ソース2
\texsource
\begin{lstlisting}
\halign{\setbox0\hbox{{$\displaystyle\int$\par}}\ht00 \dp00 \box0#1\cr
\noalign{\hrule}
www\cr
\noalign{\hrule}}
\end{lstlisting}

その結果が以下。

\noindent ソース1

\ 

\halign{\relax\setbox0\hbox{{$\displaystyle\int$\par}}\ht00 \dp00 \box0#1\cr
\noalign{\hrule}
www\cr
\noalign{\hrule}}

\ 

\noindent ソース2

\ 

\halign{\setbox0\hbox{{$\displaystyle\int$\par}}\ht00 \dp00 \box0#1\cr
\noalign{\hrule}
www\cr
\noalign{\hrule}}

\ 

\begin{question}
コンパイルもどちらも問題なく通るし,\cmd{relax}の有無の差が見えてこない。
\end{question}
\section{\LaTeX における各種レジスタの管理(大門)}

今回は,\TeX に本来備わっている\cmd{count},\cmd{dimen}など各種レジスタの,\LaTeX 流管理の実装を調べていく。

\subsection{前提知識}

\TeX は,\cmd{count}(整数),\cmd{dimen}(寸法),\cmd{skip}(グルー),\cmd{muskip}(数式用グルー),\cmd{box}(ボックス),\cmd{toks}(トークンリスト)というデータ型を持ち,素の(e-\TeX 拡張などされていない)\TeX エンジンの場合,これらの型には0から255までの番号をもつレジスタ(変数)が用意されている\footnote{その他のエンジンでは,「たくさん」(32768または65536個)のレジスタが用意されている。}。

さて,plain \TeX および(2014年以前の)\LaTeX では,この0から255まで用意された各種レジスタを,以下のように使用している。

\begin{center}
\begin{tabular}{ccccc} \hline
 番号 & \cmd{count} & \cmd{dimen},\cmd{skip} & \cmd{box} & \cmd{musikip},\cmd{toks} \\ \hline
 0 - 9 & ページ番号 & スクラッチ & スクラッチ & スクラッチ \\ \hline
 10 - 20 & 各種レジスタの割当済み番号 & 割当用 & 割当用 & 割当用 \\ \hline
 21 & \cmd{allocationnumber} & 割当用 & 割当用 & 割当用 \\ \hline
 22 & 定数``-1'' & 割当用 &割当用&割当用\\\hline
 23 - (/-1) & 割当用&割当用&割当用&割当用\\\hline
 / - 254 & insert用&insert用&insert用&割当用\\\hline
 255 &スクラッチ&スクラッチ&スクラッチ&スクラッチ\\\hline
\end{tabular}
\end{center}

ただし,/は\cmd{count20}(割当済insert番号)の値。また,「スクラッチ」は\cmd{dimen255}(\cmd{dimen@})のように,割当なしで使えるよう予め用意されたレジスタのこと

{\footnotesize『\LaTeX2$\varepsilon$【マクロ\&クラス】プログラミング基礎解説』およびマクロツイーター\par
\noindent(\texttt{http://d.hatena.ne.jp/zrbabbler/20150503/1430658474}\hspace{1zw}\texttt{http://d.hatena.ne.jp/zrbabbler/20150504/1430721128})を参照した。}


\subsection{実装コード}

\latexltx
\begin{lstlisting}[firstnumber=302]
\message{registers,}
\count10=22 % allocates \count registers 23, 24, ...
\count11=9 % allocates \dimen registers 10, 11, ...
\count12=9 % allocates \skip registers 10, 11, ...
\count13=9 % allocates \muskip registers 10, 11, ...
\count14=9 % allocates \box registers 10, 11, ...
\count15=9 % allocates \toks registers 10, 11, ...
\count16=-1 % allocates input streams 0, 1, ...
\count17=-1 % allocates output streams 0, 1, ...
\count18=3 % allocates math families 4, 5, ...
\count19=0 % allocates \language codes 1, 2, ...
\count20=255 % allocates insertions 254, 253, ...
\countdef\insc@unt=20
\countdef\allocationnumber=21
\countdef\m@ne=22 \m@ne=-1
\def\wlog{\immediate\write\m@ne}
\countdef\count@=255
\dimendef\dimen@=0
\dimendef\dimen@i=1 % global only
\dimendef\dimen@ii=2
\skipdef\skip@=0
\toksdef\toks@=0
\def\newcount{\alloc@0\count\countdef\insc@unt}
\def\newdimen{\alloc@1\dimen\dimendef\insc@unt}
\def\newskip{\alloc@2\skip\skipdef\insc@unt}
\def\newmuskip{\alloc@3\muskip\muskipdef\@cclvi}
\def\newbox{\alloc@4\box\chardef\insc@unt}
\def\newhelp#1#2{\newtoks#1#1\expandafter{\csname#2\endcsname}}
\def\newtoks{\alloc@5\toks\toksdef\@cclvi}
\def\newread{\alloc@6\read\chardef\sixt@@n}
\def\newwrite{\alloc@7\write\chardef\sixt@@n}
\def\newlanguage{\alloc@9\language\chardef\@cclvi}
\def\alloc@#1#2#3#4#5{\global\advance\count1#1\@ne
  \ch@ck#1#4#2% make sure there's still room
  \allocationnumber\count1#1%
  \global#3#5\allocationnumber
  \wlog{\string#5=\string#2\the\allocationnumber}}
\def\newinsert#1{\global\advance\insc@unt \m@ne
  \ch@ck0\insc@unt\count
  \ch@ck1\insc@unt\dimen
  \ch@ck2\insc@unt\skip
  \ch@ck4\insc@unt\box
  \allocationnumber\insc@unt
  \global\chardef#1\allocationnumber
  \wlog{\string#1=\string\insert\the\allocationnumber}}
\gdef\ch@ck#1#2#3{%
  \ifnum\count1#1<#2\else
   \errmessage{No room for a new #3}%
  \fi}
\newdimen\maxdimen \maxdimen=16383.99999pt % the largest legal <dimen>
\newskip\hideskip \hideskip=-1000pt plus 1fill % negative but can grow
\newdimen\p@ \p@=1pt % this saves macro space and time
\newdimen\z@ \z@=0pt % can be used both for 0pt and 0
\newskip\z@skip \z@skip=0pt plus0pt minus0pt
\newbox\voidb@x % permanently void box register
\end{lstlisting}

まず冒頭で,\cmd{message}によりレジスタ関連の定義が始まることを示してから処理に入る。非常に長いので,幾つかに分割してこれらのコードの動作を分析していく。

\subsection{初期設定部(仮称)について}

\begin{lstlisting}[firstnumber=302]
\message{registers,}
\count10=22 % allocates \count registers 23, 24, ...
\count11=9 % allocates \dimen registers 10, 11, ...
\count12=9 % allocates \skip registers 10, 11, ...
\count13=9 % allocates \muskip registers 10, 11, ...
\count14=9 % allocates \box registers 10, 11, ...
\count15=9 % allocates \toks registers 10, 11, ...
\count16=-1 % allocates input streams 0, 1, ...
\count17=-1 % allocates output streams 0, 1, ...
\count18=3 % allocates math families 4, 5, ...
\count19=0 % allocates \language codes 1, 2, ...
\count20=255 % allocates insertions 254, 253, ...
\countdef\insc@unt=20
\countdef\allocationnumber=21
\countdef\m@ne=22 \m@ne=-1
\def\wlog{\immediate\write\m@ne}
\countdef\count@=255
\dimendef\dimen@=0
\dimendef\dimen@i=1 % global only
\dimendef\dimen@ii=2
\skipdef\skip@=0
\toksdef\toks@=0
\end{lstlisting}

この部分は特にどうと言うことも無く,特殊な役割をもつレジスタの使用を準備しているだけである。

\subsection{\cmd{newcount}系の定義部(仮称)について}

\begin{lstlisting}[firstnumber=324]
\def\newcount{\alloc@0\count\countdef\insc@unt}
\def\newdimen{\alloc@1\dimen\dimendef\insc@unt}
\def\newskip{\alloc@2\skip\skipdef\insc@unt}
\def\newmuskip{\alloc@3\muskip\muskipdef\@cclvi}
\def\newbox{\alloc@4\box\chardef\insc@unt}
\def\newhelp#1#2{\newtoks#1#1\expandafter{\csname#2\endcsname}}
\def\newtoks{\alloc@5\toks\toksdef\@cclvi}
\def\newread{\alloc@6\read\chardef\sixt@@n}
\def\newwrite{\alloc@7\write\chardef\sixt@@n}
\def\newlanguage{\alloc@9\language\chardef\@cclvi}
\def\alloc@#1#2#3#4#5{\global\advance\count1#1\@ne
  \ch@ck#1#4#2% make sure there's still room
  \allocationnumber\count1#1%
  \global#3#5\allocationnumber
  \wlog{\string#5=\string#2\the\allocationnumber}}
\def\newinsert#1{\global\advance\insc@unt \m@ne
  \ch@ck0\insc@unt\count
  \ch@ck1\insc@unt\dimen
  \ch@ck2\insc@unt\skip
  \ch@ck4\insc@unt\box
  \allocationnumber\insc@unt
  \global\chardef#1\allocationnumber
  \wlog{\string#1=\string\insert\the\allocationnumber}}
\gdef\ch@ck#1#2#3{%
  \ifnum\count1#1<#2\else
   \errmessage{No room for a new #3}%
  \fi}
\end{lstlisting}

ここでは\cmd{newcount},\cmd{newdimen},\cmd{newskip},\cmd{newmuskip},\cmd{newbox},\cmd{newtoks}をまとめて\cmd{newcount}系と呼ぶことにする(他の\cmd{newinsert},\cmd{newhelp},\cmd{newread},\cmd{newwrite},\cmd{newlanguage}については,今回は措いておく)。これら\cmd{newcount}系の実装はどれも似通っているので,代表例として\cmd{newcount}の動作を追跡していくことにしよう。

\subsubsection{具体例}
以下の様なソースコードを実行する事を考える。
\texsource
\begin{lstlisting}
\newcount\ほげ
\end{lstlisting}
これを上に従って適宜展開・実行していくと,次のようになる。
\begin{lstlisting}
   \newcount\ほげ

-> \alloc@0\count\countdef\insc@unt\ほげ

-> \global\advance\count10\@ne
   \ch@ck0\insc@unt\count% make sure there's still room
   \allocationnumber\count10%
   \global\countdef\ほげ\allocationnumber
   \wlog{\string\ほげ=\string\count\the\allocationnumber}

   (\advanceが実行され,\count10の値が1増える)

-> \ifnum\count10<\insc@unt\else
    \errmessage{No room for a new \count}%
   \fi
   \allocationnumber\count10%
   \global\countdef\ほげ\allocationnumber
   \wlog{\string\ほげ=\string\count\the\allocationnumber}
\end{lstlisting}

ここの\cmd{ifnum}で,現在割り当てようとしているレジスタの番号が,割り当て済みのinsert番号(先の表中の``/''に相当)より小さいかを判別し,これを超過するようであればエラーが出る。このチェックを通り抜けた場合,\cmd{allocationnumber}に今から割り当てるレジスタの番号を格納し,
\begin{lstlisting}
\global\countdef\ほげ\allocationnumber
\end{lstlisting}
によってレジスタに\cmd{ほげ}を割り当てる。末尾の\cmd{wlog}は,\textit{source2e.pdf}に``Write on log file (only)''とあるように,\texttt{.log}ファイルに\cmd{ほげ}が定義された事を書き込む役割をもつ。

\subsection{その他の定義部(仮称)について}
\latexltx
\begin{lstlisting}[firstnumber=351]
\newdimen\maxdimen \maxdimen=16383.99999pt % the largest legal <dimen>
\newskip\hideskip \hideskip=-1000pt plus 1fill % negative but can grow
\newdimen\p@ \p@=1pt % this saves macro space and time
\newdimen\z@ \z@=0pt % can be used both for 0pt and 0
\newskip\z@skip \z@skip=0pt plus0pt minus0pt
\newbox\voidb@x % permanently void box register
\end{lstlisting}
\cmd{maxdimen}は寸法レジスタの最大値を設定し,\cmd{hideskip}は伸びることも可能な負のスペース(アラインメントに使用されるようだが詳細は不明)を設定する。\cmd{p@},\cmd{z@},\cmd{z@skip}はお馴染みのメモリ削減テクニックの産物である。\cmd{voidb@x}は常に「空の」boxレジスタであり,\cmd{leavevmode}の定義に利用されたり,(普通のプログラム言語におけるNull値のように)レジスタを初期化するために利用されたりする。

\end{document}
