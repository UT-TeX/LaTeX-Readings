\documentclass[autodetect-engine,dvipdfmx]{jsarticle}

\usepackage{okumacro}
\usepackage[top=25truemm,bottom=25truemm,left=25truemm,right=25truemm]{geometry}
\usepackage[dvipdfmx]{xcolor}
\usepackage{ascmac}
\usepackage{listings,jlisting}
\usepackage{latexltx2015}

\begin{document}

\title{ latex.ltxリーディング \\ 第5回資料 }
\author{ 東大\TeX 愛好会 }
\date{2015年6月12日}
\maketitle

\section{水平スペース(朝倉)}

今回は\LaTeX の水平スペースについて簡単に見ていくことにする.例によって,関連する定義を
記載する.

\latexltx
\begin{lstlisting}[firstnumber=1308]
\DeclareRobustCommand\hspace{\@ifstar\@hspacer\@hspace}
\def\@hspace#1{\hskip #1\relax}
\def\@hspacer#1{\vrule \@width\z@\nobreak
                \hskip #1\hskip \z@skip}
\newskip\fill
\fill = 0pt plus 1fill
\def\stretch#1{\z@ \@plus #1fill\relax}
\def\thinspace{\kern .16667em }
\def\negthinspace{\kern-.16667em }
\def\enspace{\kern.5em }
\def\enskip{\hskip.5em\relax}
\def\quad{\hskip1em\relax}
\def\qquad{\hskip2em\relax}
\end{lstlisting}

ほとんど\TeX のプリミティブをわかりやすく使用しているだけであり,解説するようなことは
あまりない.ひとまず,\cmd{hspace}から順に見ていくことにする.

1308行目の定義から明らかなように,\cmd{hspace}は堅い制御綴として定義されており,また\preSub
\texttt{*}\preSub の有無によって分岐が行われるようになっている.\texttt{*}\preSub がない場合には
\cmd{@hspace}が展開されて,プリミティブ\cmd{hskip}を用いた任意幅のグルーが挿入されるように
なっている.後ろに\cmd{relax}が置かれている理由は後述する.

一方,\cmd{hspace*}\preSub は一般に強制改行\preSub\verb|\\|\preSub の直後で利用されるものだが,
この場合は\cmd{@hspacer}が展開される.こちらの方が,先ほどの\cmd{@hspace}よりはやや定義が長く
なるが,いずれにせよそれほど複雑ではない.水平モードに切り替えるため幅0ptの\cmd{vrule}が挿入
されている.なお,ここに登場する\cmd{@width}は単なる文字列の置き換えマクロである.
ついでなので,\texttt{latex.ltx}の関連する部分をまとめて引っ張っておく.
\begin{lstlisting}[firstnumber=563]
\def\@height{height} \def\@depth{depth} \def\@width{width}
\def\@minus{minus}
\def\@plus{plus}
\def\hb@xt@{\hbox to}
\end{lstlisting}
こうした置換えは,\TeX の処理時にメモリ使用量を軽減するための工夫らしい.

また,\cmd{z@skip}は355行目で\preSub\verb|\newskip\z@skip \z@skip=0pt plus0pt minus0pt|と
定義されている.\cmd{newskip}は簡単に言うと寸法レジスタを生成するマクロである.後半の
\texttt{plus0pt minus0pt}はそれぞれ伸長度と収縮度を指定しており,要するに\cmd{z@skip}では
一切の伸縮を許容していないことになる.なお,\cmd{hspace}などの定義末尾に\cmd{relax}が置かれている
のはこれらの命令の後ろにたまたま``plus''などの文字列が現れた際,それが伸長度や伸縮度を指定する
意味に解釈されないようにするためである.

1313行目に登場する\texttt{fill}は特殊な単位であり,簡単に言えば「無限」の長さを表現する.
因みに\texttt{fil}や\texttt{filll}という単位も存在し,\texttt{l}の数が多いほどより強力な
無限を表現する.

\begin{question}
1311行目の\preSub\verb|\hskip \z@skip|は\cmd{relax}ではだめなのか.
\end{question}

\begin{question}
\cmd{hskip}と\cmd{kern}の使い分けはどのように行われているのか.
\end{question}




\end{document}
